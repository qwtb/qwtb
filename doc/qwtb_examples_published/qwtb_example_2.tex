\startcontents[localtoc]
\printcontents[localtoc]{}{0}{\section*{Contents}\setcounter{tocdepth}{2}}



\phantomsection
\addcontentsline{toc}{section}{Overview}
\subsection*{Overview}



Data are simulated, QWTB is used to estimate quanties using different
algorithms.



\phantomsection
\addcontentsline{toc}{section}{Generate ideal data}
\subsection*{Generate ideal data}



Sample data are generated, representing 1 second of sine waveform of nominal
frequency \texttt{fnom} 1000 Hz, nominal amplitude \texttt{Anom} 1 V and nominal phase
\texttt{phnom} 1 rad. Data are sampled at sampling frequency \texttt{fsnom} 10 kHz,
perfectly synchronized, no noise.

\begin{lstlisting}
Anom = 1; fnom = 1000; phnom = 1; fsnom = 10e4;
timestamps = [0:1/fsnom:0.1-1/fsnom];
ideal_wave = Anom*sin(2*pi*fnom*timestamps + phnom);
\end{lstlisting}


To use QWTB, data are put into two quantities: \texttt{t} and \texttt{y}. Both quantities
are put into data in structure \texttt{DI}.

\begin{lstlisting}
DI = [];
DI.t.v = timestamps;
DI.y.v = ideal_wave;
\end{lstlisting}


\phantomsection
\addcontentsline{toc}{section}{Apply three algorithms}
\subsection*{Apply three algorithms}



QWTB will be used to apply three algorithms to determine frequency and
amplitude: \texttt{SP-WFFT}, \texttt{PSFE} and \texttt{FPNLSF}. Results are in data out structure
\texttt{DOxxx}. Algorithm \texttt{FPNLSF} requires an estimate, select it to 0.1\% different
from nominal frequency. \texttt{SP-WFFT} requires sampling frequency.

\begin{lstlisting}
DI.fest.v = fnom.*1.001;
DI.fs.v = fsnom;
DOspwfft = qwtb('SP-WFFT', DI);
DOpsfe = qwtb('PSFE', DI);
DOfpnlsf = qwtb('FPNLSF', DI);
\end{lstlisting}
\begin{lstlisting}[language={},xleftmargin=5pt,frame=none]
QWTB: no uncertainty calculation
QWTB: no uncertainty calculation
QWTB: PSFE wrapper: sampling time was calculated from sampling frequency
QWTB: no uncertainty calculation
Fitting started
Fitting finished

\end{lstlisting}


\phantomsection
\addcontentsline{toc}{section}{Compare results for ideal signal}
\subsection*{Compare results for ideal signal}



Calculate relative errors in ppm for all algorithm to know which one is best.
\texttt{SP-WFFT} returns whole spectrum, so only the largest amplitude peak is
interesting. One can see for the ideal case all errors are very small.

\begin{lstlisting}
disp('SP-WFFT errors (ppm):')
[tmp, ind] = max(DOspwfft.A.v);
ferr  = (DOspwfft.f.v(ind) - fnom)/fnom .* 1e6
Aerr  = (DOspwfft.A.v(ind) - Anom)/Anom .* 1e6
pherr = (DOspwfft.ph.v(ind) - phnom)/phnom .* 1e6

disp('PSFE errors (ppm):')
ferr  = (DOpsfe.f.v - fnom)/fnom .* 1e6
Aerr  = (DOpsfe.A.v - Anom)/Anom .* 1e6
pherr = (DOpsfe.ph.v - phnom)/phnom .* 1e6

disp('FPNLSF errors (ppm):')
ferr  = (DOfpnlsf.f.v - fnom)/fnom .* 1e6
Aerr  = (DOfpnlsf.A.v - Anom)/Anom .* 1e6
pherr = (DOfpnlsf.ph.v - phnom)/phnom .* 1e6
\end{lstlisting}
\begin{lstlisting}[language={},xleftmargin=5pt,frame=none]
SP-WFFT errors (ppm):
ferr = 0
Aerr = 0
pherr = -6.5503e-09
PSFE errors (ppm):
ferr = -2.2737e-10
Aerr = -3.2196e-09
pherr = 4.7518e-08
FPNLSF errors (ppm):
ferr = -6.4733e-07
Aerr = -1.4322e-08
pherr = 8.8987e-05

\end{lstlisting}


\phantomsection
\addcontentsline{toc}{section}{Noisy signal}
\subsection*{Noisy signal}



To simulate real measurement, noise is added with normal distribution and
standard deviation \texttt{sigma} of 100 microvolt. Algorithms are again applied.

\begin{lstlisting}
sigma = 100e-6;
DI.y.v = ideal_wave + 100e-6.*randn(size(ideal_wave));
DOspwfft = qwtb('SP-WFFT', DI);
DOpsfe = qwtb('PSFE', DI);
DOfpnlsf = qwtb('FPNLSF', DI);
\end{lstlisting}
\begin{lstlisting}[language={},xleftmargin=5pt,frame=none]
QWTB: no uncertainty calculation
QWTB: no uncertainty calculation
QWTB: PSFE wrapper: sampling time was calculated from sampling frequency
QWTB: no uncertainty calculation
Fitting started
Fitting finished

\end{lstlisting}


\phantomsection
\addcontentsline{toc}{section}{Compare results for noisy signal}
\subsection*{Compare results for noisy signal}



Again relative errors are compared. One can see amplitude and phase errors
increased to several ppm, however frequency is still determined quite good by
all three algorithms. FFT is not affected by noise at all.

\begin{lstlisting}
disp('SP-WFFT errors (ppm):')
[tmp, ind] = max(DOspwfft.A.v);
ferr  = (DOspwfft.f.v(ind) - fnom)/fnom .* 1e6
Aerr  = (DOspwfft.A.v(ind) - Anom)/Anom .* 1e6
pherr = (DOspwfft.ph.v(ind) - phnom)/phnom .* 1e6

disp('PSFE errors:')
ferr  = (DOpsfe.f.v - fnom)/fnom .* 1e6
Aerr  = (DOpsfe.A.v - Anom)/Anom .* 1e6
pherr = (DOpsfe.ph.v - phnom)/phnom .* 1e6

disp('FPNLSF errors:')
ferr  = (DOfpnlsf.f.v - fnom)/fnom .* 1e6
Aerr  = (DOfpnlsf.A.v - Anom)/Anom .* 1e6
pherr = (DOfpnlsf.ph.v - phnom)/phnom .* 1e6
\end{lstlisting}
\begin{lstlisting}[language={},xleftmargin=5pt,frame=none]
SP-WFFT errors (ppm):
ferr = 0
Aerr = -0.1921
pherr = 3.5831
PSFE errors:
ferr = -5.9336e-03
Aerr = -0.1910
pherr = 5.4498
FPNLSF errors:
ferr = -2.4370e-03
Aerr = -0.1917
pherr = 4.3516

\end{lstlisting}


\phantomsection
\addcontentsline{toc}{section}{Non-coherent signal}
\subsection*{Non-coherent signal}



In real measurement coherent measurement does not exist. So in next test the
frequency of the signal differs by 20 ppm:

\begin{lstlisting}
fnc = fnom*(1 + 20e-6);
noncoh_wave = Anom*sin(2*pi*fnc*timestamps + phnom);
DI.y.v = noncoh_wave;
DOspwfft = qwtb('SP-WFFT', DI);
DOpsfe = qwtb('PSFE', DI);
DOfpnlsf = qwtb('FPNLSF', DI);
\end{lstlisting}
\begin{lstlisting}[language={},xleftmargin=5pt,frame=none]
QWTB: no uncertainty calculation
QWTB: no uncertainty calculation
QWTB: PSFE wrapper: sampling time was calculated from sampling frequency
QWTB: no uncertainty calculation
Fitting started
Fitting finished

\end{lstlisting}


\phantomsection
\addcontentsline{toc}{section}{Compare results for non-coherent signal}
\subsection*{Compare results for non-coherent signal}



Comparison of relative errors. Results of \texttt{PSFE} or \texttt{FPNLSF} are correct, however
FFT is affected by non-coherent signal considerably.

\begin{lstlisting}
disp('SP-WFFT errors (ppm):')
[tmp, ind] = max(DOspwfft.A.v);
ferr  = (DOspwfft.f.v(ind) - fnc)/fnc .* 1e6
Aerr  = (DOspwfft.A.v(ind) - Anom)/Anom .* 1e6
pherr = (DOspwfft.ph.v(ind) - phnom)/phnom .* 1e6

disp('PSFE errors:')
ferr  = (DOpsfe.f.v - fnc)/fnc .* 1e6
Aerr  = (DOpsfe.A.v - Anom)/Anom .* 1e6
pherr = (DOpsfe.ph.v - phnom)/phnom .* 1e6

disp('FPNLSF errors:')
ferr  = (DOfpnlsf.f.v - fnc)/fnc .* 1e6
Aerr  = (DOfpnlsf.A.v - Anom)/Anom .* 1e6
pherr = (DOfpnlsf.ph.v - phnom)/phnom .* 1e6
\end{lstlisting}
\begin{lstlisting}[language={},xleftmargin=5pt,frame=none]
SP-WFFT errors (ppm):
ferr = -20.000
Aerr = -2.8780
pherr = 6291.9
PSFE errors:
ferr = -1.1368e-10
Aerr = 6.6613e-10
pherr = 2.9754e-08
FPNLSF errors:
ferr = -6.0583e-07
Aerr = -1.5765e-08
pherr = 8.3299e-05

\end{lstlisting}


\phantomsection
\addcontentsline{toc}{section}{Harmonically distorted signal.}
\subsection*{Harmonically distorted signal.}



In other cases a harmonic distortion can appear. Suppose a signal with second
order harmonic of 10\% amplitude as the main signal.

\begin{lstlisting}
hadist_wave = Anom*sin(2*pi*fnom*timestamps + phnom) + 0.1*Anom*sin(2*pi*fnom*2*timestamps + 2);
DI.y.v = hadist_wave;
DOspwfft = qwtb('SP-WFFT', DI);
DOpsfe = qwtb('PSFE', DI);
DOfpnlsf = qwtb('FPNLSF', DI);
\end{lstlisting}
\begin{lstlisting}[language={},xleftmargin=5pt,frame=none]
QWTB: no uncertainty calculation
QWTB: no uncertainty calculation
QWTB: PSFE wrapper: sampling time was calculated from sampling frequency
QWTB: no uncertainty calculation
Fitting started
Fitting finished

\end{lstlisting}


\phantomsection
\addcontentsline{toc}{section}{Compare results for harmonically distorted signal.}
\subsection*{Compare results for harmonically distorted signal.}



Comparison of relative errors. \texttt{SP-WFFT} or \texttt{PSFE} are not affected by harmonic
distortion, however \texttt{FPNLSF} is thus is not suitable for such signal.

\begin{lstlisting}
disp('SP-WFFT errors (ppm):')
[tmp, ind] = max(DOspwfft.A.v);
ferr  = (DOspwfft.f.v(ind) - fnom)/fnom .* 1e6
Aerr  = (DOspwfft.A.v(ind) - Anom)/Anom .* 1e6
pherr = (DOspwfft.ph.v(ind) - phnom)/phnom .* 1e6

disp('PSFE errors:')
ferr  = (DOpsfe.f.v - fnom)/fnom .* 1e6
Aerr  = (DOpsfe.A.v - Anom)/Anom .* 1e6
pherr = (DOpsfe.ph.v - phnom)/phnom .* 1e6

disp('FPNLSF errors:')
ferr  = (DOfpnlsf.f.v - fnom)/fnom .* 1e6
Aerr  = (DOfpnlsf.A.v - Anom)/Anom .* 1e6
pherr = (DOfpnlsf.ph.v - phnom)/phnom .* 1e6
\end{lstlisting}
\begin{lstlisting}[language={},xleftmargin=5pt,frame=none]
SP-WFFT errors (ppm):
ferr = 0
Aerr = 0
pherr = -6.6613e-09
PSFE errors:
ferr = -2.2737e-10
Aerr = -4.3299e-09
pherr = 4.4853e-08
FPNLSF errors:
ferr = -0.7820
Aerr = 0.1103
pherr = 236.67

\end{lstlisting}


\phantomsection
\addcontentsline{toc}{section}{Harmonically distorted, noisy, non-coherent signal.}
\subsection*{Harmonically distorted, noisy, non-coherent signal.}



In final test all distortions are put in a waveform and results are compared.

\begin{lstlisting}
err_wave = Anom*sin(2*pi*fnc*timestamps + phnom) + 0.1*Anom*sin(2*pi*fnc*2*timestamps + 2) + 100e-6.*randn(size(ideal_wave));
DI.y.v = err_wave;
DOspwfft = qwtb('SP-WFFT', DI);
DOpsfe = qwtb('PSFE', DI);
DOfpnlsf = qwtb('FPNLSF', DI);
\end{lstlisting}
\begin{lstlisting}[language={},xleftmargin=5pt,frame=none]
QWTB: no uncertainty calculation
QWTB: no uncertainty calculation
QWTB: PSFE wrapper: sampling time was calculated from sampling frequency
QWTB: no uncertainty calculation
Fitting started
Fitting finished

\end{lstlisting}


\phantomsection
\addcontentsline{toc}{section}{Compare results for harmonically distorted, noisy, non-coherent signal.}
\subsection*{Compare results for harmonically distorted, noisy, non-coherent signal.}

\begin{lstlisting}
disp('SP-WFFT errors (ppm):')
[tmp, ind] = max(DOspwfft.A.v);
ferr  = (DOspwfft.f.v(ind) - fnc)/fnc .* 1e6
Aerr  = (DOspwfft.A.v(ind) - Anom)/Anom .* 1e6
pherr = (DOspwfft.ph.v(ind) - phnom)/phnom .* 1e6

disp('PSFE errors:')
ferr  = (DOpsfe.f.v - fnc)/fnc .* 1e6
Aerr  = (DOpsfe.A.v - Anom)/Anom .* 1e6
pherr = (DOpsfe.ph.v - phnom)/phnom .* 1e6

disp('FPNLSF errors:')
ferr  = (DOfpnlsf.f.v - fnc)/fnc .* 1e6
Aerr  = (DOfpnlsf.A.v - Anom)/Anom .* 1e6
pherr = (DOfpnlsf.ph.v - phnom)/phnom .* 1e6
\end{lstlisting}
\begin{lstlisting}[language={},xleftmargin=5pt,frame=none]
SP-WFFT errors (ppm):
ferr = -20.000
Aerr = 0.3775
pherr = 6295.2
PSFE errors:
ferr = 1.1917e-04
Aerr = 2.7827
pherr = 1.8512
FPNLSF errors:
ferr = -0.7626
Aerr = 2.8970
pherr = 233.03

\end{lstlisting}


\stopcontents[localtoc]
