\begin{tightdesc}
\item [Id:] SFDR
\item [Name:] Spurious Free Dynamic Range
\item [Description:] Calculates Spurious Free Dynamic Range of a signal based on an amplitude spectrum.
\item [Citation:] Implementation: Martin Sira
\item [Remarks:] Samples are expected in quantity `y`, and algorithm `SP-WFFT` is used to calculate the amplitude spectrum. Alternatively a spectrum can be directly delivered in quantity `A`, use of blackman DFT window is expected.
\item [License:] MIT
\item [Provides GUF:] no
\item [Provides MCM:] no
\item [Input Quantities] \rule{0em}{0em}
    \begin{tightdesc}
    \item [Required:] 
        \textsf{y} or \textsf{A}
    \item [Descriptions:] \rule{0em}{0em}
        \begin{tightdesc}
            \item[\textsf{A}] -- Amplitude spectrum
            \item[\textsf{y}] -- Sampled values
        \end{tightdesc}
    \end{tightdesc}
\item [Output Quantities:] \rule{0em}{0em}
    \begin{tightdesc}
        \item[\textsf{SFDR}] -- Spurious Free Dynamic Range, relative to carrier (V/V)
        \item[\textsf{SFDRdBc}] -- Spurious Free Dynamic Range, relative to carrier, in decibel (dB)
    \end{tightdesc}
\end{tightdesc}
