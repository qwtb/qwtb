\startcontents[localtoc]
\printcontents[localtoc]{}{0}{\subsection*{Contents}\setcounter{tocdepth}{2}}



\phantomsection
\addcontentsline{toc}{section}{Generate sample data}
\subsubsection*{Generate sample data}



Two quantities are prepared: \texttt{t} and \texttt{y}, representing 1 second of sinus waveform of nominal
frequency 1 kHz, nominal amplitude 1 V, nominal phase 1 rad and offset 1 V sampled at sampling
frequency 10 kHz.

\begin{lstlisting}
DI = [];
Anom = 2; fnom = 100; phnom = 1; Onom = 0.2;
DI.t.v = [0:1/1e4:1-1/1e4];
DI.y.v = Anom*sin(2*pi*fnom*DI.t.v + phnom) + Onom;
\end{lstlisting}


Add a noise with normal distribution probability:

\begin{lstlisting}
noisestd = 1e-4;
DI.y.v = DI.y.v + noisestd.*randn(size(DI.y.v));
\end{lstlisting}


Lets make an estimate of frequency 0.2 percent higher than nominal value:

\begin{lstlisting}
DI.fest.v = 100.2;
\end{lstlisting}


\phantomsection
\addcontentsline{toc}{section}{Calculate estimates of signal parameters}
\subsubsection*{Calculate estimates of signal parameters}



Use QWTB to apply algorithm \texttt{FPNLSF} to data \texttt{DI}.

\begin{lstlisting}
CS.verbose = 1;
DO = qwtb('FPNLSF', DI, CS);
\end{lstlisting}
\begin{lstlisting}[language={},xleftmargin=5pt,frame=none]
QWTB: no uncertainty calculation
Fitting started
Fitting finished

\end{lstlisting}


\phantomsection
\addcontentsline{toc}{section}{Copy results to inputs}
\subsubsection*{Copy results to inputs}



Take results of \texttt{FPNLSF} and put them as inputs \texttt{DI}.

\begin{lstlisting}
DI.f = DO.f;
DI.A = DO.A;
DI.ph = DO.ph;
DI.O = DO.O;
\end{lstlisting}


Suppose the signal was sampled by a 20 bit digitizer with full scale range \texttt{FSR} of 6 V (+- 3V). (The
signal is not quantised, so the quantization noise is not present. Thus the simulation and results
are not fully correct.):

\begin{lstlisting}
DI.bitres.v = 20;
DI.FSR.v = 3;
\end{lstlisting}


\phantomsection
\addcontentsline{toc}{section}{Calculate SINAD and ENOB}
\subsubsection*{Calculate SINAD and ENOB}

\begin{lstlisting}
DO = qwtb('SINAD-ENOB', DI, CS);
\end{lstlisting}
\begin{lstlisting}[language={},xleftmargin=5pt,frame=none]
QWTB: no uncertainty calculation

\end{lstlisting}


\phantomsection
\addcontentsline{toc}{section}{Display results:}
\subsubsection*{Display results:}



Results are:

\begin{lstlisting}
SINADdB = DO.SINADdB.v
ENOB = DO.ENOB.v
\end{lstlisting}
\begin{lstlisting}[language={},xleftmargin=5pt,frame=none]
SINADdB = 82.936
ENOB = 13.068

\end{lstlisting}


Theoretical value of SINADdB is 20*log10(Anom./(noisestd.*sqrt(2))). Theoretical value of ENOB is
log2(DI.range.v./(noisestd.*sqrt(12))). Absolute error of results are:

\begin{lstlisting}
SINADdBtheor = 20*log10(Anom./(noisestd.*sqrt(2)));
ENOBtheor = log2(DI.FSR.v./(noisestd.*sqrt(12)));
SINADerror = SINADdB - SINADdBtheor
ENOBerror = ENOB - ENOBtheor
\end{lstlisting}
\begin{lstlisting}[language={},xleftmargin=5pt,frame=none]
SINADerror = -0.074709
ENOBerror = -0.012410

\end{lstlisting}


\stopcontents[localtoc]
