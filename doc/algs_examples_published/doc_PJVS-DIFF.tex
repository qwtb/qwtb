\startcontents[localtoc]
\printcontents[localtoc]{}{0}{\subsection*{Contents}\setcounter{tocdepth}{2}}



\phantomsection
\addcontentsline{toc}{section}{Generate sample data}
\subsubsection*{Generate sample data}



First quantity \texttt{y} representing 1 second of signal containing spurious component is prepared. Main
signal component has nominal frequency 1 kHz, nominal amplitude 2 V, nominal phase 1 rad and
offset 1 V sampled at sampling frequency 10 kHz.

\begin{lstlisting}
DI = [];
fsnom = 1e4; Anom = 4; fnom = 100; phnom = 1; Onom = 0.2;
t = [0:1/fsnom:1-1/fsnom];
DI.y.v = Anom*sin(2*pi*fnom*t + phnom);
\end{lstlisting}


A spurious component with amplitude at 1/100 of main carrier frequency is added. Thus by
definition the SFDR in dBc has to be 40.

\begin{lstlisting}
DI.y.v = DI.y.v + Anom./100*sin(2*pi*fnom*3.5*t + phnom);
\end{lstlisting}


\phantomsection
\addcontentsline{toc}{section}{Call algorithm}
\subsubsection*{Call algorithm}



Use QWTB to apply algorithm \texttt{SFDR} to data \texttt{DI}.

\begin{lstlisting}
DO = qwtb('SFDR', DI);
\end{lstlisting}
\begin{lstlisting}[language={},xleftmargin=5pt,frame=none]
QWTB: no uncertainty calculation
QWTB: no uncertainty calculation

\end{lstlisting}


\phantomsection
\addcontentsline{toc}{section}{Display results}
\subsubsection*{Display results}



Result is the SFDR (dBc).

\begin{lstlisting}
SFDR = DO.SFDRdBc.v
\end{lstlisting}
\begin{lstlisting}[language={},xleftmargin=5pt,frame=none]
SFDR = 40.000

\end{lstlisting}


\stopcontents[localtoc]
