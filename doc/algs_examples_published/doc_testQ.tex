
% This LaTeX was auto-generated from an M-file by MATLAB.
% To make changes, update the M-file and republish this document.

%%% \documentclass{article}
%%% \usepackage{graphicx}
%%% \usepackage{color}

%%% \sloppy
%%% \definecolor{lightgray}{gray}{0.5}
\setlength{\parindent}{0pt}

%%% \begin{document}

    
    
\subsection*{testQ}

\begin{par}
Example for algorithm testQ. Algorithm is usefull only for testing QWTB toolbox. It calculates maximal and minimal value of the record.
\end{par} \vspace{1em}
\begin{par}
See also \lstinline{qwtb}
\end{par} \vspace{1em}

\subsubsection*{Contents}

\begin{itemize}
\setlength{\itemsep}{-1ex}
   \item Generate sample data
   \item Call algorithm
   \item Different input
\end{itemize}


\subsubsection*{Generate sample data}

\begin{par}
Quantities are prepared.
\end{par} \vspace{1em}
\begin{lstlisting}[style=mcode]
DI = [];
DI.x.v = [-5:-1:-1];
DI.v.v = [1:10];
DI.a.v = 5;
DI.c.v = 'variable c';
\end{lstlisting}


\subsubsection*{Call algorithm}

\begin{par}
Use QWTB to apply algorithm \lstinline{testQ} to data \lstinline{DI}.
\end{par} \vspace{1em}
\begin{lstlisting}[style=mcode]
DO = qwtb('testQ', DI);
\end{lstlisting}

        \begin{lstlisting}[style=output]
QWTB: no uncertainty calculation
\end{lstlisting} \color{black}
    \begin{par}
The result is maximal value of \lstinline{v}, i.e. 10:
\end{par} \vspace{1em}
\begin{lstlisting}[style=mcode]
DO.max.v
\end{lstlisting}

        \begin{lstlisting}[style=output]

ans =

    10

\end{lstlisting} \color{black}
    

\subsubsection*{Different input}

\begin{par}
Add quantity \lstinline{u}, which has precedence over \lstinline{v}:
\end{par} \vspace{1em}
\begin{lstlisting}[style=mcode]
DI.u.v = [100:110];
DO = qwtb('testQ', DI);
\end{lstlisting}

        \begin{lstlisting}[style=output]
QWTB: no uncertainty calculation
\end{lstlisting} \color{black}
    \begin{par}
The result is maximal value of \lstinline{u}, i.e. 110:
\end{par} \vspace{1em}
\begin{lstlisting}[style=mcode]
DO.max.v
\end{lstlisting}

        \begin{lstlisting}[style=output]

ans =

   110

\end{lstlisting} \color{black}
    


%%% \end{document}
    
