\startcontents[localtoc]
\printcontents[localtoc]{}{0}{\subsection*{Contents}\setcounter{tocdepth}{2}}



\phantomsection
\addcontentsline{toc}{section}{Generate sample data}
\subsubsection*{Generate sample data}



Quantities are prepared.

\begin{lstlisting}
DI = [];
DI.x.v = [-5:-1:-1];
DI.v.v = [1:10];
DI.a.v = 5;
DI.c.v = 'variable c';
\end{lstlisting}


\phantomsection
\addcontentsline{toc}{section}{Call algorithm}
\subsubsection*{Call algorithm}



Use QWTB to apply algorithm \texttt{testQ} to data \texttt{DI}.

\begin{lstlisting}
DO = qwtb('testQ', DI);
\end{lstlisting}
\begin{lstlisting}[language={},xleftmargin=5pt,frame=none]
QWTB: no uncertainty calculation

\end{lstlisting}


The result is value of \texttt{v}:

\begin{lstlisting}
DO.e.v
\end{lstlisting}
\begin{lstlisting}[language={},xleftmargin=5pt,frame=none]
ans = [](0x0)

\end{lstlisting}


\phantomsection
\addcontentsline{toc}{section}{Different input}
\subsubsection*{Different input}



Add quantity \texttt{u}, which has precedence over \texttt{v}:

\begin{lstlisting}
DI.u.v = [100:110];
DO = qwtb('testQ', DI);
\end{lstlisting}
\begin{lstlisting}[language={},xleftmargin=5pt,frame=none]
QWTB: no uncertainty calculation

\end{lstlisting}


The result is value of \texttt{u}:

\begin{lstlisting}
DO.e.v
\end{lstlisting}
\begin{lstlisting}[language={},xleftmargin=5pt,frame=none]
ans = [](0x0)

\end{lstlisting}


\stopcontents[localtoc]
