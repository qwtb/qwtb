\startcontents[localtoc]
\printcontents[localtoc]{}{0}{\subsection*{Contents}\setcounter{tocdepth}{2}}



\phantomsection
\addcontentsline{toc}{section}{Generate sample data}
\subsubsection*{Generate sample data}



Two quantities are prepared: \texttt{t} and \texttt{y}, representing 1 second of sinus waveform of nominal
frequency 1 kHz, nominal amplitude 1 V, nominal phase 1 rad and offset 1 V sampled at sampling
frequency 10 kHz.

\begin{lstlisting}
DI = [];
Anom = 2; fnom = 100; phnom = 1; Onom = 0.2;
DI.t.v = [0:1/1e4:1-1/1e4];
DI.y.v = Anom*sin(2*pi*fnom*DI.t.v + phnom) + Onom;
\end{lstlisting}


Lets make an estimate of frequency 0.2 percent higher than nominal value:

\begin{lstlisting}
DI.fest.v = 100.2;
\end{lstlisting}


\phantomsection
\addcontentsline{toc}{section}{Call algorithm}
\subsubsection*{Call algorithm}



Use QWTB to apply algorithm \texttt{FPNLSF} to data \texttt{DI}.

\begin{lstlisting}
CS.verbose = 1;
DO = qwtb('FPNLSF', DI, CS);
\end{lstlisting}
\begin{lstlisting}[language={},xleftmargin=5pt,frame=none]
QWTB: no uncertainty calculation
Fitting started
Fitting finished

\end{lstlisting}


\phantomsection
\addcontentsline{toc}{section}{Display results}
\subsubsection*{Display results}



Results is the amplitude, frequency and phase of sampled waveform.

\begin{lstlisting}
A = DO.A.v
f = DO.f.v
ph = DO.ph.v
O = DO.O.v
\end{lstlisting}
\begin{lstlisting}[language={},xleftmargin=5pt,frame=none]
A = 2.0000
f = 100.000
ph = 1.0000
O = 0.2000

\end{lstlisting}


Errors of estimation in parts per milion:

\begin{lstlisting}
Aerrppm = (DO.A.v - Anom)/Anom .* 1e6
ferrppm = (DO.f.v - fnom)/fnom .* 1e6
pherrppm = (DO.ph.v - phnom)/phnom .* 1e6
Oerrppm = (DO.O.v - Onom)/Onom .* 1e6
\end{lstlisting}
\begin{lstlisting}[language={},xleftmargin=5pt,frame=none]
Aerrppm = -7.7716e-10
ferrppm = -4.0927e-08
pherrppm = 5.6228e-06
Oerrppm = -4.1217e-08

\end{lstlisting}


\stopcontents[localtoc]
